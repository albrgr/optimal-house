\documentclass[12pt,final,fleqn]{article}

% basic packages
\usepackage[margin=1in] { geometry }
\usepackage{amssymb,amsmath, bm}
\usepackage{verbatim}
\usepackage[latin1]{inputenc}
%\usepackage[OT1]{fontenc}
\usepackage{setspace}
\usepackage{natbib}
\usepackage{enumitem}
\usepackage{url}
\usepackage[font={bf}]{caption}
%\usepackage{pgfplots}
%\usepackage[font={bf}]{caption}
\usepackage{setspace}
\usepackage{latexsym}
%\usepackage{euscript}
\usepackage{graphicx}
\usepackage{marvosym}
%\usepackage[varg]{txfonts}  Older version of ``g'' in math.

% bibliography packages
\usepackage{natbib}
\bibpunct{(}{)}{;}{a}{}{,}
\bibliographystyle{apsr}
\renewcommand{\bibname}{References}

% hyperref options
\usepackage{color}
\usepackage{hyperref}
\usepackage{xcolor}
\hypersetup{
    colorlinks,
    linkcolor={blue!50!black},
    citecolor={blue!50!black},
    urlcolor={blue!80!black}
}
\newcommand*{\Appendixautorefname}{Appendix}
\renewcommand*{\sectionautorefname}{Section}
\renewcommand*{\subsectionautorefname}{Section}
\newcommand{\aref}[1]{\hyperref[#1]{Appendix~\ref{#1}}}

% packages for tables
\usepackage{longtable}
\usepackage{booktabs, threeparttable}
\usepackage{threeparttablex}
%\usepackage{tabularx}
% dcolumn package
\usepackage{dcolumn}
\newcolumntype{.}{D{.}{.}{-1}}
\newcolumntype{d}[1]{D{.}{.}{#1}}
\captionsetup{belowskip=10pt,aboveskip=-5pt}
\usepackage{multirow}
% rotating package
\usepackage[figuresright]{rotating}
\usepackage{pdflscape}
\usepackage{subcaption}

% packages for figures
\usepackage{grffile}
\usepackage{afterpage}
\usepackage{float}
\usepackage[section]{placeins}

% theorem package
\usepackage{theorem}
\theoremstyle{plain}
\theoremheaderfont{\scshape}
\newtheorem{theorem}{Theorem}
\newtheorem{algorithm}{Algorithm}
\newtheorem{assumption}{Assumption}
\newtheorem{lemma}{Lemma}
\newtheorem{proposition}{Proposition}
\newtheorem{remark}{Remark}
\newcommand{\qed}{\hfill \ensuremath{\Box}}
\newcommand\indep{\protect\mathpalette{\protect\independenT}{\perp}}
\DeclareMathOperator{\sgn}{sgn}
\DeclareMathOperator{\tr}{tr}
\DeclareMathOperator{\argmin}{arg\min}
\DeclareMathOperator{\argmax}{arg\max}
\def\independenT#1#2{\mathrel{\rlap{$#1#2$}\mkern2mu{#1#2}}}
\providecommand{\norm}[1]{\lVert#1\rVert}
\renewcommand\r{\right}
\renewcommand\l{\left}
\newcommand\E{\mathbb{E}}
\newcommand\dist{\buildrel\rm d\over\sim}
\newcommand\iid{\stackrel{\rm i.i.d.}{\sim}}
\newcommand\ind{\stackrel{\rm indep.}{\sim}}
\newcommand\cov{{\rm Cov}}
\newcommand\var{{\rm Var}}
\newcommand\SD{{\rm SD}}
\newcommand\bone{\mathbf{1}}
\newcommand\bzero{\mathbf{0}}

% dotted lines in tables
%\usepackage{arydshln}

\usepackage{pdflscape}

% spacing between sections and subsections
\usepackage[compact]{titlesec}

% times new roman
%\usepackage{times}

% appendix settings
\usepackage[toc,page,header]{appendix}
\renewcommand{\appendixpagename}{\centering Appendices}
\usepackage{chngcntr}
\usepackage{etoolbox}
\usepackage{lipsum}

% online appendix settings
\renewcommand{\figurename}{Supplemental Figure}
\renewcommand{\tablename}{Supplemental Table}
\renewcommand{\tableautorefname}{Supplemental Table}
\renewcommand{\figureautorefname}{Supplemental Figure}
\renewcommand{\sectionautorefname}{Appendix Section}
\renewcommand{\subsectionautorefname}{Appendix Section}
\newcommand{\apref}[1]{\hyperref[#1]{Appendix~\ref{#1}}}
\usepackage{xr}
\externaldocument{optimal_house}

% file paths and definitions
\input{../output/ch1txtstats.txt}
\makeatletter
\newcommand*\ExpandableInput[1]{\@@input#1 }
\makeatother

\setlength{\mathindent}{1cm}
\allowdisplaybreaks[4]
\doublespacing
%\special{pdf: pagesize width 8.5truein height 11.0truein}

\titleformat{\subsection}
  {\itshape\large}{\thesubsection}{1em}{}

\begin{document}
\author{Devin Incerti}
\title{\textbf{Supplemental Online Appendix}}
\date{\today}
\maketitle

\tableofcontents
\listoffigures
\listoftables

\clearpage
\doublespacing

\section{Data}
The analyses in this paper requires data for three separate analyses. First, foreceasting variables are needed for the Bayesian hierarchical model. Second, both district and national polls are used for the Bayesian DLM. Third, campaign data is used to examine whether political parties donate in a manner consistent with the theoretical model. More information about these variables are provided in Appendix Sections~\ref{sec:forecast vars}-\ref{sec:campaign spending}. \autoref{sec:data sources} provides information on the sources used to collect the data.

\subsection{Forecasting Variables} \label{sec:forecast vars}
The regression forecast includes variables (and error terms) at the national and district level.\footnote{Regional variables were also considered but they did not improve the fit of the model and only increased model complexity.} Forecasts are made during August of each election year using post 1980 data so all variables are measured before September 1st.

The primary dependent variable is the Democratic share of the two-party vote. This is not the only possible choice for the response variable but it is frequently used in the literature. Other choice that are essentially identical include the incumbent party's share of the vote or the margin of victory for the incumbent candidate.

House elections are highly persistent from one election to the next so they can be predicted quite accurately using only the lag of the two-party vote. The lag is unavailable without substantial reaggregation in redistricting years so, as is common in the literature, the analysis excludes years ending in 2.\footnote{Examples of other studies that deal with redistricting years in this manner include \citet{gelman1990estimating}, \citet{kastellec2008predicting}, and \citet{gelman2008estimating}.} The lagged vote is tied in many respects to individual candidates and is not surprisingly a much stronger predictor of the vote when incumbents are running for re-election than in open-seats \citep{gelman2008estimating}. The model consequently includes the lag of the presidential vote in each district in addition to the lag of the House vote. The presidential vote is subtracted from the nationwide presidential vote to control for national trends and can be interpreted as a measure of the degree to which a district leans toward one party or another.\footnote{This variable is very similar to the Cook Partisan Voting Index (Cook PVI), which compares the two-party vote in the past two presidential elections to the nation's average share of the same presidential vote. The main difference is that I only include the previous presidential election becase of redistricting.}\footnote{Models that interacted both the lagged vote and the centered presidential vote with whether incumbents were running were also considered, but did not improve the fit of the model.}

One of the largest and most consistent findings in the political science literature on American elections is that, all else equal, incumbent candidates receive more votes than challengers \citep{{gelman1990estimating, lee2001electoral, ansolabehere2002incumbency, gelman2008estimating}}. To model this incumbency effect, I use the variable, \emph{Incumbent}, which is equal to $1$ if the incumbent is a Democrat, $-1$ if the incumbent is a Republican and $0$ if there is no incumbent running. $+1, 0, -1$ variables are used repeatedly in the analysis and are always equal to +1 for Democrats and -1 for Republicans; they should be thought of as dummy variables that are constrained to have the same impact on the vote for both Democratic and Republican candidates (recall that the dependent variable is the Democratic candidate's share of the two-party vote).

Since more experienced candidates tend to do better at the polls, I include two $+1, 0, -1$ experience variables: \emph{Freshman incumbent} and \emph{Previous office holder}. \emph{Freshman incumbent} is equal to +1(-1) if a Democratic (Republican) incumbent was elected for the first time in the previous election and 0 otherwise. \emph{Previous office holder} is equal to +1 or -1 if a Democrat or Republican challenger had previously held office and was running against either an incumbent or a challenger without previous experience; it is equal to 0 in all other situations including open seats in which both candidates had previous political experience.

The model also accounts for the ideology of candidates relative to the ideology of voters in their districts. Candidate ideology is measured using the first dimension DW-nominate score---which measures the political ideology of candidates and ranges on a liberal-conservative scale from -1 to 1. A measure of district ideology on the same scale as the DW-nominate score was created by taking a weighted average of the DW-nominate scores of the most recent presidential candidates for each party, with weights equal to each party's share of the district two-party vote in the most recent presidential election.\footnote{Michael Dukakis does not have a DW-nominate score so his ideology was imputed using the mean score among Democratic presidential candidates between 1980 and 2010.} The variable used in the model, \emph{Relative 1st dimension DW-nominate score}, is the DW-nominate score minus the district ideology score.\footnote{This variable outperformed another variable which interacted the absolute value of the DW-nominate with the party controlling the seat, suggesting that voting moderation is more strongly associated with higher vote shares than a candidate's ideological distance from the ideology of his or her voters.} In incumbent districts, this variable will have a positive sign if candidates do better at the polls when they have more moderate voting records (perhaps because they appeal more to independent voters), conditional on winning the primary election. In open seats, this variable will be positive if voters tend to elect a Democratic candidate when the previous candidate(s) had a conservative voting record and a Republican candidate when the previous candidate(s) had a liberal voting record.\footnote{In open districts, the DW-nominate score is equal to the average DW-nominate score among all representatives in the previous Congress. Most districts only have one representative per congressional session but certain events such as deaths create scenarios in which this is not the case.}\footnote{In OLS regressions, the coefficient on the interaction of the relative DW-nominate score variable with whether a seat is open is essentially zero, implying that the effect of the variable does not differ by whether a seat is open or not.}

The model also includes three nationwide variables: the president's average August approval rating, the August generic ballot, and an indicator variable for whether the election is being held in a midterm year.\footnote{A variable including second quarter GDP was also considered but it was not statistically or economically significant.} The \emph{August generic ballot} variable is an average of generic ballot polls in the month of August in each election year. Both the variable, \emph{midterm election} and the variable, \emph{August presidential net approval rating} are multiplied by +1 if the president is a Democrat and -1 if Republican. The impetus for \emph{midterm election} is previous research showing that the party of the president usually loses seats during midterm election \citep[e.g.][]{erikson1988puzzle}. 

Summary statistics for these variables are presented in~\autoref{table:Summary Statistics for Forecasting Variables}. Variables rarely exceed one in absolute value and are thus on a similar scale to the dependent variable.

\subsection{Polling Data}
A steady stream of polls asking about the generic congressional vote during the 2010 election were released starting as early as December of 2008. All told, the website \href{http://www.realclearpolitics.com/}{Real Clear Politics} gathered  $\gbcount$ polls published by $\countgbpollsters$ different polling organizations over $\gbdays$ days with the last poll conducted $\final_gbday$ days prior to the November 2nd election.\footnote{Polls are typically conducted over multiple days. I consider the poll date to be the midpoint between the start and end date of polling (when the poll is conducted over an even number of days I use the median day closest to the start day).} Most pollsters only published a few polls (median $= \gbpollstermedian$), but the two most prolific pollsters,  Rasmussen and Gallup, conducted $\gbpollsterstop\%$ of all polls. The median sample size across surveys was $\gbmedian_samplesize$ and the average percentage of undecided respondents was $\gbundecided\%$.

District polls were considerably less prolific. Polls were published by $\countpollsters$ different pollsters for $\pollcdcount$ out of the 435 Congressional districts and the median number of polls per district in districts with at least 1 poll was $\pollmedianpercd$. The median sample size for these polls was $\pollmediansamplesize$. 

\autoref{fig:Summary Plots for District Polls} provides summaries of the frequencies at which district polls were fielded. As shown in panel (a) of the figure, polls were considerably more prevalent in close races (those in which the final vote share was close to $0.5$), although few to no polls were published in some close races. This is important for forecasting the election because it means that the districts that are the most difficult to predict are the ones that have the most polling data. This matters for optimal spending as well since it reduces uncertainty about the closeness of a race and provides parties with more information about which districts to target when attempting to influence the the election.

Panel (b) shows the cumulative number of polls during each day of the campaign. Nearly 2/3 of the polls were fielded after the start of September, when the rate of polling began accelerating quickly. This temporal pattern is consistent with the monthly pattern in contributions (see \autoref{sec:campaign spending}), suggesting that the final two months of campaigns capture the most crucial moments.

\subsection{Campaign Spending} \label{sec:campaign spending}
House candidates receive funds from three primary sources: party committees, PACs, and individuals.\footnote{Organizations---such as firms, unions or trade associations---that wish to spend money to influence federal elections must create a separate source of funds known as a PAC.} Campaign donors can, in turn, spend on the behalf of candidates by either making a direct contribution, a coordinated expenditure, or an indirect expenditure for or against a candidate.\footnote{A common example of an independent expenditure is a TV advertisement praising a candidate or criticizing an opponent. Before 2010, corporations and unions could only fund independent expenditures from their PACs; however in 2010, the U.S. Supreme Court rule in \emph{Citizens United v. Federal Election Commission} that corporations and unions could use their own treasuries to raise money for independent expenditures.} Independent expenditures are not limited by law but cannot be coordinated with campaigns. By contrast, direct contributions and coordinated expenditures can be coordinated with a campaign but are limited by federal law.\footnote{Coordinate expenditures, are, as the name suggests, expenditures made on the behalf of campaigns that can be discussed with the campaign.}

Although campaign fund-raising is a year-round endeavor for political candidates, it follows a cyclical pattern and increases as election day nears. This is shown in~\autoref{fig:Average Daily Expenditures by PACs on Candidates by Month, 2000 - 2010}, which plots average daily spending by PACs in election years by month across the 2000 - 2010 elections. Expenditures are fairly constant from January until August and then jump considerably in September before peaking at around \$4 million dollars per day in October and then decreasing slightly in November.\footnote{November expenditures only include spending that occurred prior to election day. Between 2000 - 2010, the latest election occurred on November 7th.} 

Political parties are the focus of this paper, so it is worth examining the spending patterns of party committees in detail. The official campaign arms of the Democratic and Republican party in the House are the Democratic Congressional Committee (DCCC) and the National Republican Congressional Committee (NRCC). The principal goal of these organizations is to assist House candidates and help elect them to Congress. Direct contributions and coordinated expenditure are severely limited by contributions limits, so almost all spending is made independently of candidates. The Democratic National Committee (DNC) and Republican National Committee (RNC) sometimes spend money on House candidates as well, but the amount is very small compared to the amount spent by the DCCC and NRCC.

While the national party committees have clear incentives to maximize the electoral success of their respective parties, it is unclear whether other organizations share this same goal. \citet{herrnson2009roles} has argued that the reach of political parties extends considerably further than with the national party committees. He views parties as multilayered coalitions that include actors not usually considered as components of the party. I adopt this classification scheme and separate parties into three separate categories: national party committees, party-connected committees, and allied PACs. 

The national party committees are the DCCC, NRCC, DNC and RNC. Party-connected committees consist of state and local party committees, candidate committees, and House candidate's leadership PACs. These committees have incentives to adopt spending strategies based on private or local goals, but \citet{herrnson2009roles} argues that they attempt to advance their party's collective goals as well. Spending by state and local party committees is also tied to the desires of the national parties because a significant portion of their receipts come from transfers from the DNC and RNC.\footnote{The state and local party committees receive most of their funds from individual contributions and transfers from affiliated party committees.} Finally, following \citet{herrnson2009roles}, I consider allied PACs to be PACs that spend over $90\%$ of their funds on one party's candidates.

Post-August spending for these groups and other non-aligned PACs are shown in~\autoref{table:Average Post-August Spending by Political Parties per District in Non-Redistricting Years, 2000 - 2010}. The table reports spending per district averaged over all non-redistricting year elections from 2000 - 2010. The national party committees spend considerably more than all other groups as national party committee spending is between 45 and 50 percent of all PAC and party spending for both the Democratic and Republican parties. Allied PACs and other non-aligned PACs are the next largest spenders. PACs are considerably more likely to be allied with Democrats than Republicans, although spending by non-aligned PACs is similar for both Republican and Democratic candidates. Spending by party-connected committees makes up less than $10\%$ of total PAC and party committee spending, making them the least prolific spending group. In total, slightly more is spent on Democrats than Republicans.

Candidates also receive a large percentage of their funds from individuals. Contributions from individuals donating over \$200 are itemized but contributions from smaller donors are not. During the 2009 - 2010 election cycle, only $0.26\%$ of the U.S. population gave more than \$200, but total contributions for these individuals summed to \$1.9 billion while contributions from all other individuals were only \$1.1 billion.\footnote{See \url{https://www.opensecrets.org/bigpicture/}.} Individuals who donated over \$200 spent \$$\expindivsdem$ and \$$\expindivsrep$ per district on Democratic and Republican candidates after August 31st in non-redistricting years from 2000 - 2010 respectively. Since contributions for individuals who donated less than \$200 are not itemized, I cannot make a comparable calculation for donations from these individuals. However, if one assumes that spending on House candidates mirrors total spending, then contributions from small donors should have totaled around \$58,000.\footnote{Individuals donating less than \$200 spent $\$1.1\; \rm{billion}/\$1.9\; \rm{billion} \approx 0.58$ as much as larger donors in 2010. These larger donors, in turn, spend around \$100,000 per district so smaller donors likely spent close to $0.58 \cdot \$100,000 \approx \$58,000$.} Total PAC and party committee spending on House candidates is therefore larger than spending from individuals and the national party committees are the largest single source of funds for House candidates.

\subsection{Data Sources}\label{sec:data sources}
\subsubsection{Gary Jacobson Data}
Data on House elections from 1946 - 2012 were generously provided by Gary Jacobson. The Gary Jacobson data provide district level information including the Democrat's share of the two party vote in both House and presidential elections, whether an incumbent is running for reelection (and the party of the incumbent), and whether a challenger has previously held office.

\subsubsection{DW-Nominate Score}
The data is available for download at \url{http://www.voteview.com/dwnominate.asp}. The DW-nominate scores are the first dimension scores originally developed by Keith Poole and Howard Rosenthal, which can be interpreted as the liberal-conservative divide in modern politics \citep{poole1997congress, poole2011ideology}.

\subsubsection{Congressional Committee Data}
Data on the committee assignments of members of Congress from each district are from \citep{stewart2015}.

\subsubsection{National Variables}
\emph{Second quarter GDP}: The percent change in U.S. GDP from one quarter to the next in chained 2009 dollars was considered but not used in the final hierarchical model. It is available for download from the Bureau of Economic Analysis at \url{http://www.bea.gov/national/index.htm}. \\
\emph{Presidential net approval rating}: Approval ratings were scraped from president specific webpages with available at \url{http://www.ropercenter.uconn.edu/CFIDE/roper/presidential/webroot/presidential_rating.cfm}. The net approval ratings is the percentage of survey respondents who approve of the president minus the percentage who disapprove. The ratings are based on polls conducted by various polling organizations multiple times each month. \\
\emph{Generic congressional ballot:} Gallup polls were searched for and downloaded using the iPOLL Databank at \url{http://www.ropercenter.uconn.edu/data_access/ipoll/ipoll.html}. There were two main survey questions of interest that were typically worded as ``If the elections for Congress were being held today, which party's candidate would you vote for in your 
Congressional district'' and ``As of today, do you lean more toward''. I consider a respondent to favor a particular party if they say they would vote for that party or they lean more toward that party. Although a sample of likely voters is preferred, Gallup does not include the data used to estimate a likely voter sample, so I use a registered voter sample instead. Survey results for this question are more difficult to obtain than presidential approval ratings since each poll must be searched for individually so I consequently restrict the generic ballot questions to those asked by Gallup in August (the month the forecast is being made).

\subsubsection{Spending Data}
Data on campaign spending are from the campaign finance data provided by the Center for Responsive Politics (CRP) (\url{https://www.opensecrets.org/}). CRP obtains the campaign finance data from the Federal Election Commission and adds value to it by cleaning and categorizing the data. Spending to specific candidates is from the ``PAC'' table and the ``indivs'' table which contain itemized spending data for PACs and individuals contributing over \$200 respectively.\footnote{Contributions from party committees are included in the PAC table.} The data can be separated by date, individual employer, and PAC type. Information on the districts of specific candidates is linked to these tables with the ``Cands'' table, which provides information on each candidate. 

\subsubsection{House Polls}
The district polls are all of those used by the New York times to forecast the 2010 House election and  were scraped from \url{http://elections.nytimes.com/2010/forecasts/house}. National polls were scraped from the website \url{http://www.realclearpolitics.com/} and include all polls listed for the generic congressional vote. Each poll contains the date that the poll was taken, its sample size, and the proportion of respondents favoring the Democrat and Republican candidate. 

\section{Gibbs Sampler for the DLM} \label{sec: Gibbs}
\subsection{DLM Form} \label{sec: DLM Form}
\subsubsection{National Model} 
Let $K_t$ be the numbers of polls published on day $t$ and $J$ be the number of polling firms. Then, equations~\ref{observation eqn national} and \ref{system eqn national} from the main text can be written in the form,
\begin{align}
\label{eqn:natdlm_obs}
\underset{(K_t \times 1)}{Y_t} &= \underset{(K_t \times 1)}{F_t} \underset{(1 \times 1)}{\theta_t} + \underset{(K_t \times J)}{A_t} \underset{(J \times 1)}{\Lambda} + \underset{(K_t \times 1)}{v_t},\; v_t \sim N(0, V_t)\\
\label{eqn:natdlm_state}
\underset{(1 \times 1)}{\theta_t} &= \underset{(1 \times 1)}{G}\underset{(1 \times 1)}{\theta_{t-1}} + \underset{(1 \times 1)}{w_t},\; w_t \sim N(0, W),
\end{align}
where $Y_t$ is a column vector of all polls published at time $t$, $F_t$ is a column vector of $1$'s, $A_t$ is a matrix of binary indicators equal to $1$ if polling firm $j$ published the $k$'th poll and $0$ otherwise, $G=1$, $\theta_t = \mu_t$, $\Lambda =[\lambda_1\; \lambda_2\cdots \lambda_J]^T$,  $V_t$ is a $K_t \times K_t$ diagonal matrix with $\sigma^2_k$ along the diagonal and $W=\sigma^2_\mu$. Appendix equation~\ref{eqn:natdlm_obs} is known as the observation equation while Appendix equation~\ref{eqn:natdlm_state} is known as the state equation. 

\subsubsection{District Model} 
Let $m$ be the number of districts with polling data and $L_t$ be the numbers of polls published on day $t$. Then, equations~\ref{observation eqn district} and \ref{system eqn district} can be written in the form,
\begin{align}
\label{eqn: dlm observation}
\underset{(L_t \times 1)}{Y_t} &= \underset{(L_t \times m)}{F_t} \underset{(m \times 1)}{\theta_t} + \underset{(L_t \times 1)}{v_t},\; v_t \sim N(0, V_t)\\
\label{eqn: dlm state}
\underset{(m \times 1)}{\theta_t} &= \underset{(m \times m)}{G}\underset{(m \times 1)}{\theta_{t-1}} + \underset{(m \times 1)}{w_t},\; w_t \sim N(0, W),
\end{align}
where $Y_t$ is a column vector of all district polls less national opinion at time $t$, $F_t$ is a matrix of binary indicators equal to 1 if poll $l$ is from district $m$, $G= I_m$, $\theta_t = [\xi_{1t}\; \xi_{2t}\cdots \xi_{mt}]^T$, $V_t$ is an $L_t \times L_t$ diagonal matrix with $\sigma^2_l$ along the diagonal and $W = \sigma^2_\xi$.

\subsection{Gibbs Sampler}
\subsubsection{National Model}
Using the DLM form, the joint posterior density of the states and variance parameters is,
\begin{align}
p(\theta_{1:T}, \sigma_\mu^2, \Lambda|Y_{1:T}, A_t, V_{1:T})&=\prod_{t=1}^T p(Y_t|\theta_t, \Lambda, A_t, V_t) \prod_{t=1}^T p(\theta_t|\theta_{t-1}, \sigma_\mu^2) \nonumber\\
&\qquad \times p(\theta_0)p(\sigma_\mu^2)p(\Lambda).
\end{align}
The variance parameter, $\sigma^2_\mu$, is given an inverse gamma prior, $p(\sigma^2_\mu)\sim\rm{IG}(\alpha_\mu, \beta_\mu)$ and the house effects, $\Lambda$, are given independent normal priors, $p(\lambda_j) \sim N(0, \sigma^2_\lambda)$. Using these priors and recalling that $\theta_t = \mu_t$ and $G_t =1$, the Gibbs sampler then proceeds as follows by drawing from the conditional distributions of the unknown parameters.
\begin{enumerate}
\item $\sigma_\mu^2$. The conditional distribution is,
\begin{align}
p(\sigma_\mu^2|\ldots)&\propto p(\sigma_\mu^2) \prod_{t=1}^T p(\theta_{t}|\theta_{t-1}, \sigma_\mu^2) \\
&\propto (\sigma_\mu^2)^{(\alpha_\mu - 1)}\exp\left[\frac{\beta_\mu}{\sigma_\mu^2}\right](\sigma_\mu^2)^{T/2} \prod_{t=1}^T \exp\left[\frac{(\mu_{t} - \mu_{t-1})^2}{2\sigma_\mu^2}\right]\\
&\propto \sigma_\mu^{2(\alpha_\mu + T/2 - 1)}\exp\left\{\frac{1}{\sigma_\mu^2}\left[\beta_\mu + \frac{1}{2}\sum_{t=1}^T(\mu_{t} - \mu_{t-1})^2\right]\right\},
\end{align}
which is an inverse gamma distribution,
\begin{align}
p(\sigma_\mu^2|\ldots)&\sim \mathcal{IG}\left(\alpha_\mu + T/2, \beta_\mu + \frac{1}{2}\sum_{t=1}^T (\mu_{t} - \mu_{t-1})^2  \right).
\end{align}
\item $\lambda_j$. The conditional distributions for the House effects are normally distributed since both the prior and the likelihood are normal. Letting $\mathcal{P}_j$ refer to the set of polls published by firm $j$ during the campaign and using standard results for conjugate normal models, $\lambda_j$ is normal with mean,
\begin{align}
\left[\sum_{k \in \mathcal{P}_j}\frac{y_k - \mu_{t[k]}}{\sigma^2_k}\right] \left[\sum_{k \in \mathcal{P}_j} \frac{1}{\sigma^2_k} + \frac{1}{\sigma^2_\lambda}\right]^{-1}
\end{align}
and variance,
\begin{align}
\left[\sum_{k \in \mathcal{P}_j} \frac{1}{\sigma^2_k} + \frac{1}{\sigma^2_\lambda}\right]^{-1}.
\end{align}
\item $\theta_{1:T}$. The states can be sampled from their conditional distribution,
\begin{align}
\label{eqn: conditional state}
p(\theta_{1:T}|\ldots)&= p(\theta_T| Y_{1:T}, A_t, \Lambda, \sigma^2_\mu) \prod_{t=0}^{T-1} p(\theta_t|\theta_{t+1}, Y_{1:t}, A_t, \Lambda, \sigma^2_\mu).
\end{align}
Equation~\ref{eqn: conditional state} suggests that if we could sample $\theta_T$ then we could recursively backwards sample through the rest of the states, $\theta_{1:T-1}$. The \emph{forward filtering backwards sampling} (FFBS) is the standard method for generating this sample. The filtering portion of the algorithm is the Kalman filter discussed in the text and works by starting from an initial value $\theta_0$ and then sequentially updating each $\theta_t$ as new information becomes available. The backwards sampling part then smooths the values of the states by first sampling from $\theta_T$, and then from the conditional distribution of $\theta_{T-1}$, and so on all the way through all of the remaining states $\theta_{1:T-2}$. To summarize, the FFBS algorithm proceeds in the following manner: 
\begin{enumerate}
\item Run Kalman filter.
\item Draw $\theta_T \sim N(m_T, C_T)$ where $m_T = E(\theta_T|Y_{1:T}, A_t, \Lambda, \sigma^2_\mu)$ and $C_T = \var(\theta_T | Y_{1:T}, A_t, \Lambda, \sigma^2_\mu)$.
\item For $T-1, \ldots, 0$ draw $\theta_t \sim N(h_t, H_T)$ where $h_t = E(\theta_t|\theta_{t+1}, Y_{1:t}, A_t, \Lambda, \sigma^2_\mu)$ and $H_t = \var(\theta_t|\theta_{t+1}, Y_{1:t}, A_t, \Lambda, \sigma^2_\mu)$.
\end{enumerate}
In this case, letting $\mathcal{P}_t$ refers to the set of all polls published on day $t$ we have,
\begin{align}
m_t &= \left[\sum_{k \in \mathcal{P}_t} \frac{y_k - \lambda_{j[k]}}{\sigma^2_k} + \frac{M_{t-1}}{C_{t-1} + \sigma^2_\mu}\right]\cdot C_T,\\
C_t &= \left[\sum_{k \in \mathcal{P}_t} \frac{1}{\sigma^2_k} + \frac{1}{C_{t-1} + \sigma^2_\mu} \right]^{-1},
\end{align}
and,
\begin{align}
h_t &= \left[\frac{\theta_{t+1}}{\sigma^2_\mu}+ \frac{m_t}{C_t}\right]H_t, \\
H_t &= \left[\frac{1}{\sigma^2_\mu} + \frac{1}{C_t}\right]^{-1}.
\end{align}
For additional details on the Kalman filter and the smoothing algorithm see a text such as \citet{west1997bayesian}.
\end{enumerate}

\subsubsection{District Model}
Inference for the district model proceeds in a similar manner to the national model using the DLM form from equations~\ref{eqn: dlm observation} and \ref{eqn: dlm state}. The posterior density is,
\begin{align}
p(\theta_{1:T}, \sigma^2_{\xi}|Y_{1:t}) &= \prod_{t=1}^T p(Y_t|\theta_t, V_t)\prod_{t=1}^T p(\theta_t|\theta_{t-1}, \sigma^2_\xi) p(\theta_0) p(\sigma^2_\xi).
\end{align}
Again using an inverse gamma distribution for the variance parameter, $\sigma^2_\xi \sim \rm{IG}(\alpha_\xi, \beta_\xi)$, the Gibbs sampler is as follows.
\begin{enumerate}
\item $\sigma_\xi^2$. Draw from the conditional distribution,
\begin{align}
p(\sigma_\xi^2|\ldots)&\propto p(\sigma_\xi^2) \prod_{t=1}^T p(\theta_{t}|\theta_{t-1}, \sigma_\xi^2) \\
&\propto (\sigma_\xi^2)^{(\alpha_\xi - 1)}\exp\left[\frac{\beta_\xi}{\sigma_\xi^2}\right](\sigma_\xi^2)^{Tm/2} \prod_{t=1}^T \prod_{i=1}^m \exp\left[\frac{(\xi_{it} - \xi_{i,t-1})^2}{2\sigma_\xi^2}\right]\\
&\propto \sigma_\xi^{2(\alpha_\xi + Tm/2 - 1)}\exp\left\{\frac{1}{\sigma_\xi^2}\left[\beta_\xi + \frac{1}{2}\sum_{t=1}^T \sum_{i=1}^m (\xi_{it} - \xi_{i,t-1})^2\right]\right\},
\end{align}
which is an inverse gamma distribution,
\begin{align}
p(\sigma_\xi^2|\ldots)&\sim \mathcal{IG}\left(\alpha_\xi + Tm/2, \beta_\xi + \frac{1}{2}\sum_{t=1}^T \sum_{i=1}^m (\xi_{it} - \xi_{i,t-1})^2  \right).
\end{align}
\item $\theta_{1:T}$. The states $\xi_{it}$ are drawn using the FFBS algorithm.
\end{enumerate}

\section{Forecasting the Two-Party Vote}
\subsection{Bayesian Hierarchical Model}
Posterior means and quantiles for the parameters using data from 1980 - 2008 (which was used to forecast the 2010 election) are shown in~\autoref{table:Posterior Inferences for Parameters in Forecasting Model, House Elections 1980 - 2008}. All of the variables (except the August generic ballot) are precisely estimated and have 95\% credibility intervals that do not include zero. The posterior densities for all parameters are all approximately normally distributed,  which is consistent with the central limit theorem in a Bayesian context (see~\autoref{fig:Posterior Densities for Parameters in Forecasting Model} in~\aref{sec: Additional Figures 1}).\footnote{The distribution on the national error is somewhat of an exception because it is somewhat right-skewed. This likely occurs because the number of national elections is small and the the error terms are truncated at zero.}

The signs of the variables are generally as expected. For example, the president's party does worse in midterm years, challengers with previous political experience receive more votes and voters tend to prefer representatives with less extreme DW-nominate scores. The sign on the freshman incumbent variable may be somewhat of a surprise though, as less experienced incumbents are actually predicted to do better than their more experienced counterparts. Finally, the estimated incumbency effect is consistent with previous research, although it is a little closer to typical estimates of the incumbency effect in recent years (of around $10\%$) than the more involved method used by \citet{gelman2008estimating} (which found the incumbency effect to be closer to $8\%$).\footnote{\citet{gelman2008estimating} estimate candidate level incumbency effects using a multilevel model. They also include a model for the party of the incumbent which depends on the district vote lagged two periods.}

Although parameters have only been reported for the model using data from the 1980 to 2008 elections, the model was actually fit 5 separate times in order to imitate 5 separate real-time forecasts using data that would have been available to a political analyst in August \emph{prior} to each election. Each model has its own unique posterior distribution for the parameters. In practice, the district parameters do not vary much from one election to the next but that national parameters (which are estimated using much less data) become more precise over time. 

\subsection{Bayesian DLM}
\subsubsection{Posterior distribution of variance terms}
Estimates of the square root of the variance parameters, $\sigma_\mu$ and $\sigma_\xi$ are shown in~\autoref{table:Posterior Quantiles for Square Root of DLM Variance Paramters}. The table reports estimates listed by the date the model was estimated. The posterior quantiles for $\sigma_\mu$ are consistent across dates and yield $95\%$ credible intervals ranging from around $0.015$ to  $0.023$ with a median of $0.018$. Posterior estimates of $\sigma_\xi$ are more variable but are generally pretty stable; the estimates do however increase slightly as election day nears.

\subsubsection{DLM Forecast Evaluation}
I evaluated the performance of the DLM using two commonly used summary measures. The first summary measure is the standard root mean square forecast error (RMSFE) described in the main body of the text. Here the standard DLM (with a prior) does considerably better than DLM with no prior. Both however have considerably lower RMSFE's than the regression based forecast from the hierarchical model. Strangely, the RMSFE's do not decrease during the campaign.

The second summary measure is the Brier score, which is often used to evaluate the accuracy of binary probabilistic forecasts. For this paper, the Brier score at time $t$ can be written as $BS_t = \frac{1}{408} \sum_{i=1}^{408}(f_{it} - a_{iT})^2$, where $408$ is the number of contested House seats; $f_{it}$ is the forecasted probability that the Democratic candidate would win district $i$ at time $t$, and $a_T$ is a binary variable equal to $1$ if the Democratic candidate won the actual seat and $0$ if the Republican candidate won. Higher Brier scores indicate worse forecasting performance. As shown in the figure, the Brier scores for both DLM's decline in a linear fashion over time and are considerably smaller than the Brier score of the hierarchical model. In contrast to the RMSFE, the Brier sore suggests that the DLM without a prior is a better predictor than the DLM with a prior. 

The results suggest that forecasts with the DLM's are more accurate than forecast with the hierarchical model alone, but it is somewhat disconcerting that the RMSFEs increase as the election day gets closer. \autoref{fig:RMSFE By Polling Availability and Forecast Date} in the appendix shows that this is being caused by districts with no polling data: the RMSFEs decrease (increase) over time in districts with (without) at least one poll. The poor performance of the RMSFEs in districts without polls is an artifact of the downward trend in the forecast of the national Democratic vote (see~\autoref{fig:Forecasts of the 2010 National Vote by Date}). This is illustrated in appendix~\autoref{fig:MPE By Polling Availability and Forecast Date}, which plots the mean prediction error (MPE)---defined as $E[f_{it}-v_{iT}]$ where $v_{iT}$ is the observed vote on election day---by forecast date. The plot shows that forecasts become increasingly biased toward Republicans and that the bias is only severe in districts without polls (since the poll based forecasts in districts without polls are driven entirely by the biased national polls). The bias increases the overall RMSFE but not the overall Brier score because the Brier score is more sensitive to changes in forecasts of close elections (which are the most likely to have polling data) than lopsided ones while the RMSFE is equally sensitive to changes in both.

\section{Relationship Between $\mathbf{Q_i}$ and District Spending}
\subsection{Alternative Measure of Electoral Competitiveness}
The finding that campaign contributions are higher in close elections is a common result in the political science and economics literatures. As mentioned earlier, one drawback of these studies is that they use the ex-post electoral margin as a measure of the closeness of the election and assume a linear relationship between this margin and spending.\footnote{Some studies used the electoral margin from the previous election as a measure of closeness.} Do the values of $Q_i$ calculated using Stromberg's probabilistic voting model fit the data better?

\autoref{fig:Correlation Between Measures of Electoral Competitiveness and Campaign Spending} takes a look at this question by plotting the correlation between various measure of electoral competitiveness. Based on~\autoref{fig:Correlation Between Q and Spending, by Contributor Type}, I use $Q_i^{seats}$ as the measure of electoral competitiveness derived from the probabilistic voting model. The electoral margin is measured as the absolute value of the Democratic vote share less 0.5.\footnote{Uncontested seats are treated as having an electoral margin equal to $0.5$ to maintain consistency with the probabilistic voting model.} $Q_i^{seats}$ fits the data considerably better than the electoral margin for all five elections in the sample. There is also little difference between calculating electoral margin with the actual election outcome or a forecasted value, implying that $Q_i^{seats}$ accounts for observed spending better because it depends on forecast uncertainty and that the normality assumptions on the error terms are reasonable. 

\section{Supplemental Tables}
\begin{table}[!ht]
\footnotesize
\begin{center}
\begin{threeparttable}
\caption{Summary Statistics for Forecasting Variables} \label{table:Summary Statistics for Forecasting Variables}
\begin{tabular*}{\textwidth}{l@{\extracolsep{\fill}}d{-1}d{-1}d{-1}}
\vspace{-5pt}\\
\hline
\hline
\multicolumn{1}{c}{Variable} & \multicolumn{1}{c}{Min} & \multicolumn{1}{c}{Median} & \multicolumn{1}{c}{Max}   \\
\hline
\ExpandableInput{../tables/sumstats.txt}
\hline
\hline
\end{tabular*}
\scriptsize
Notes: \emph{Relative Dem. share of presidential vote in last election} is the deviation of the Democratic share of the presidential vote in each district from the national vote in the most recent presidential election. \emph{Incumbent} is equal to +1 or -1 depending on whether a Democrat or Republican is running for re-election and 0 otherwise. \emph{Relative 1st dimension DW-nominate score} is the 1st dimension DW-nominate score minus a measure of district ideology. \emph{Freshman incumbent} is defined in the same was as \emph{Incumbent} but is only equal to +1 or -1 if the incumbent was elected for the first time in the previous election. \emph{Previous office-holder} is equal to +1 (-1) if a non-incumbent Democrat (Republican) had previously held office and 0 otherwise (or if the seat was open and both candidates had previously held office). \emph{August presidential net approval} is the president's average net approval rating in August multiplied by +1 if the current president is a Democrat and -1 if Republican. \emph{August generic ballot} is an average of generic ballot polls in the month of August. \emph{Midterm election} is equal to 0 in non-midterm years and +1 (-1) if the president is a Democrat (Republican) and it is a midterm election year. Other variables such as second quarter GDP growth and the party currently controlling a district were also considered, but were not included in the final model because they were not statistically significant and they did not improve the fit of the model.
\end{threeparttable}
\end{center}
\end{table}

\begin{table}[!ht]
\footnotesize
\begin{center}
\begin{threeparttable}
\caption{Average Post-August Spending by Political Parties per District in Non-Redistricting Years, 2000 - 2010} \label{table:Average Post-August Spending by Political Parties per District in Non-Redistricting Years, 2000 - 2010}
\begin{tabular*}{\textwidth}{l@{\extracolsep{\fill}}d{-1}d{-1}d{-1}d{-1}}
\vspace{-5pt}\\
\hline
\hline
\multicolumn{1}{c}{} & \multicolumn{2}{c}{Democratic party} & \multicolumn{2}{c}{Republican party}\\
\cmidrule{2-3} \cmidrule{4-5}
\multicolumn{1}{c}{} & \multicolumn{1}{c}{\$} & \multicolumn{1}{c}{Fraction of total}& \multicolumn{1}{c}{\$}& \multicolumn{1}{c}{Fraction of total}  \\
\hline
\ExpandableInput{../tables/partyexp.txt}
\hline
\hline
\end{tabular*}
\scriptsize
Notes: 2002 is omitted because it is a redistricting year. Spending consists of direct contributions, coordinated expenditures and independent expenditures for a canditate or against the opposing candidate in 2010 dollars. Spending figures are per district means averaged across the 2000 to 2010 non-redistricting year elections. Expenditures prior to September 1st are excluded. National party committees include the DNC, DCCC, RNC and RCCC. Party-connected committee are candidate committees or leadership PACs. Allied PAC consist of PACs that have spent 90\% or more of their funds on one political party. Other PACs are all other PACs that do not fit into the other three categories.
\end{threeparttable}
\end{center}
\end{table}

\begin{table}[!ht]
\footnotesize
\begin{center}
\begin{threeparttable}
\caption{Posterior Inferences for Parameters in Forecasting Model, House Elections 1980 - 2008} \label{table:Posterior Inferences for Parameters in Forecasting Model, House Elections 1980 - 2008}
\begin{tabular*}{\textwidth}{l@{\extracolsep{\fill}}d{-1}d{-1}d{-1}d{-1}}
\vspace{-5pt}\\
\hline
\hline
\multicolumn{2}{c}{} & \multicolumn{3}{c}{Posterior quantiles}\\
\cmidrule{3-5}
\multicolumn{1}{c}{Variable} & \multicolumn{1}{c}{Mean} & \multicolumn{1}{c}{2.5\%}& \multicolumn{1}{c}{Median}& \multicolumn{1}{c}{97.5\%}  \\
\hline
\ExpandableInput{../tables/postsummary.txt}
\hline
\hline
\end{tabular*}
\scriptsize
\end{threeparttable}
\end{center}
\end{table}

\begin{table}[!ht]
\footnotesize
\begin{center}
\begin{threeparttable}
\caption{Posterior Quantiles for Square Root of DLM Variance Paramters} \label{table:Posterior Quantiles for Square Root of DLM Variance Paramters}
\begin{tabular*}{\textwidth}{l@{\extracolsep{\fill}}d{-1}d{-1}d{-1}d{-1}d{-1}d{-1}d{-1}}
\vspace{-5pt}\\
\hline
\hline
\multicolumn{1}{c}{} & \multicolumn{3}{c}{National model ($\sigma_\mu$)}& \multicolumn{3}{c}{District model ($\sigma_\xi$)}\\
\cmidrule{2-4} \cmidrule{5-7}
\multicolumn{1}{c}{Months to election} & \multicolumn{1}{c}{2.5\%}& \multicolumn{1}{c}{Median}& \multicolumn{1}{c}{97.5\%}  & \multicolumn{1}{c}{2.5\%}& \multicolumn{1}{c}{Median}& \multicolumn{1}{c}{97.5\%}  \\
\hline
\ExpandableInput{../tables/postpsi.txt}
\hline
\hline
\end{tabular*}
\scriptsize Notes: Months to election refers to the date that the model was estimated.
\end{threeparttable}
\end{center}
\end{table}

%\renewcommand\thefigure{\thesection.\arabic{figure}}    
\section{Supplemental Figures} \label{sec: Additional Figures 1}
\setcounter{figure}{0}


\begin{figure}[!htb]
\centering
\includegraphics{../figs/polldescstats.pdf}
\vspace{.5cm}
\caption{Summary Plots for District Polls}
\label{fig:Summary Plots for District Polls}
\end{figure} 

 \begin{figure}[!htb]
\centering
\includegraphics{../figs/monthly_spending.pdf}
\vspace{.5cm}
\caption{Average Daily Expenditures by PACs on Candidates by Month, 2000 - 2010}
\label{fig:Average Daily Expenditures by PACs on Candidates by Month, 2000 - 2010}
\begin{minipage}{\linewidth}
\footnotesize
\emph{Notes:} Expenditures are the average (across elections from 2000 to 2010) of the sum of expenditures on candidates by all PACS by month. Expenditures in non-election years are not included. Spending is in millions of 2010 dollars. November spending only include expenditures prior to election day. 
\end{minipage}
\end{figure}


\begin{landscape}
\begin{figure}[!ht]
\includegraphics{../figs/postden.pdf}
\vspace{.5cm}
\caption{Posterior Densities for Parameters in Forecasting Model}
\label{fig:Posterior Densities for Parameters in Forecasting Model}
\begin{minipage}{\linewidth}
\footnotesize
\emph{Notes:} Variables are the same as in~\autoref{table:Summary Statistics} but names have been shortened for graphical purposes. 
\end{minipage}
\end{figure}
\end{landscape}

\iffalse
\begin{figure}[!htb]
\includegraphics{../figs/pva.pdf}
\vspace{.5cm}
\caption{In-Sample Predicted Versus Actual Plot, House Elections 1980 - 2008}
\label{fig:In-Sample Predicted Versus Actual Plot, House Elections 1980 - 2008}
\begin{minipage}{\linewidth}
\footnotesize
\emph{Notes:} Predicted value is the mean of the in-sample posterior predictive distribution ($v^{rep}$). Model parameters are estimated using data from 1980 to 2008. The blue line is a 45 degree line.
\end{minipage}
\end{figure}
\fi

\begin{figure}[!htb]
\includegraphics{../figs/pvaoos.pdf}
\vspace{.5cm}
\caption{Out-of-Sample Predicted Versus Actual Plot, House Elections 1990 - 2010}
\label{fig:Out-of-Sample Predicted Versus Actual Plot, House Elections 1990 - 2010}
\begin{minipage}{\linewidth}
\footnotesize
\emph{Notes:} Predicted value is the mean of the out-of-sample posterior predictive distribution. Model parameters are estimated using data from 1980 up until the year the prediction is being made (i.e. 1980 - 2008 when predicting the 2010 election). The blue line is a 45 degree line. 
\end{minipage}
\end{figure}

\begin{figure}[!htb]
\centering
\includegraphics{../figs/natopinion1.pdf}
\vspace{.5cm}
\caption{Estimates of National Opinion During the Election Campaign}
\label{fig:Estimates of National Opinion During the Election Campaign}
\begin{minipage}{\linewidth}
\footnotesize
\emph{Notes:} Estimates are the mean of the posterior distribution of the states at each date from a DLM using 2010 generic ballot polling data up until election day.
\end{minipage}
\end{figure}

\begin{figure}[!htb]
\begin{center}
\includegraphics{../figs/houeffects.pdf}
\vspace{.5cm}
\caption{Estimates of House Effects}
\label{fig:Estimates of House Effects}
\begin{minipage}{\linewidth}
\footnotesize
\emph{Notes:} House effects are estimated using the DLM with 2010 generic ballot polling data up until election day. The bias across pollsters is assumed to sum to zero for identification purposes. The reported biases are from the late-October model just prior to election day.
\end{minipage}
\end{center}
\end{figure}

\begin{figure}[!htb]
\begin{center}
\includegraphics{../figs/rmsfe2010_bypollfreq.pdf}
\vspace{.5cm}
\caption{RMSFE By Polling Availability and Forecast Date}
\label{fig:RMSFE By Polling Availability and Forecast Date}
\begin{minipage}{\linewidth}
\footnotesize
%\emph{Notes:}
\end{minipage}
\end{center}
\end{figure}

\begin{figure}[!htb]
\begin{center}
\includegraphics{../figs/mpe2010_bypollfreq.pdf}
\vspace{.5cm}
\caption{MPE By Polling Availability and Forecast Date}
\label{fig:MPE By Polling Availability and Forecast Date}
\begin{minipage}{\linewidth}
\footnotesize
\emph{Notes:} The MPE is the mean of the forecasted Democratic share of the two-party vote minus the Democratic share of the two-party vote in the actual election. Negative value reflect a bias toward the Republican party.
\end{minipage}
\end{center}
\end{figure}

\begin{figure}[!htb]
\centering
\includegraphics{../figs/forecast_eval_2010.pdf}
\vspace{.5cm}
\caption{Evaluations of 2010 Election Forecasts}
\label{fig:Evaluations of 2010 Election Forecasts}
\begin{minipage}{\linewidth}
\footnotesize
\emph{Notes:} Figures (a) and (b) plot the root mean square forecast errors and brier scores respectively for the different forecast models by forecast date. 
\end{minipage}
\end{figure}

\begin{figure}[!htb]
\begin{center}
\includegraphics{../figs/brier2010_bypollfreq.pdf}
\vspace{.5cm}
\caption{Brier Score By Polling Availability and Forecast Date}
\label{fig:Brier Score By Polling Availability and Forecast Date}
\begin{minipage}{\linewidth}
\footnotesize
%\emph{Notes:}
\end{minipage}
\end{center}
\end{figure}

\begin{figure}[!htb]
\includegraphics{../figs/corbyparty.pdf}
\vspace{.5cm}
\caption{Correlation Between Q and PAC Spending, by Party}
\label{fig:Correlation Between Q and PAC Spending, by Party}
\begin{minipage}{\linewidth}
\footnotesize
\emph{Notes:} PAC spending for a party consists of a direct contribution by a PAC to that party or an indirect contribution by a PAC against the opposing party. The top panel shows the correlation between $Q_i^{maj}$ and spending while the bottom panel shows the correlation between $Q_i^{seats}$ and spending.
\end{minipage}
\end{figure}

\begin{figure}[!htb]
\includegraphics{../figs/corbymeasure.pdf}
\vspace{.5cm}
\caption{Correlation Between Measures of Electoral Competitiveness and Campaign Spending}
\label{fig:Correlation Between Measures of Electoral Competitiveness and Campaign Spending}
\begin{minipage}{\linewidth}
\footnotesize
\emph{Notes:} Margin is the absolute value of the actual (forecasted) two-party vote share in a district minus 0.5. Campaign spending is the sum of all spending by PACs, party committees and individuals contributing over \$200 in a given district and year.
\end{minipage}
\end{figure}





\pdfbookmark[1]{References}{References}
\bibliography{bibliography}

\end{document} 
